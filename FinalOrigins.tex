\documentclass[8 pt, leqno]{article}
\usepackage[english]{babel}
\usepackage[utf8]{inputenc}
\usepackage{fancyhdr}
\usepackage{wrapfig}
\usepackage[margin=.75in]{geometry}
\usepackage{amssymb,amsmath,amsthm,mathrsfs}
\usepackage{graphicx}
\usepackage{caption}
\usepackage{pgfplots}
\usepackage[utf8]{inputenc}
\usepackage[T1]{fontenc}
\pgfplotsset{
  compat=1.9,
  unit code/.code 2 args={\si{#1#2}} % from manual, for using siunitx to typeset units
}
\usepackage{mdframed}
\usepackage{inputenc}
\usepackage{siunitx}
\usepackage{siunitx}
\usepgfplotslibrary{groupplots,units}
\newlength\figureheight 
\newlength\figurewidth 
\setlength\figureheight{0.4\textwidth}
\setlength\figurewidth{0.35\textwidth}
\usepackage{tikz}
\setcounter{section}{2}
\usepackage[hidelinks]{hyperref}
\usepackage{graphicx}
\usepackage{subcaption}
\usepackage{multicol}
 
\pagestyle{fancy}
\fancyhf{}
\fancyhead[CO]{A Novel Approach To Modeling Complex Systems with Brownian Motion}
\linespread{1.5}

%\title{}
%\date{}
%\author{}
\begin{document}
%\maketitle

Modeling complex systems, from the growth of cancer cells and bacteria to the evolution of a species, is critical to our understanding of the physical world and our place in it. To comprehend our origins, it is essential to accurately model the dynamics driving the creation of life, and to comprehend our future, it is essential to have accurate predictive models. Such modeling is complicated when the dynamics behind these complex systems is random, or stochastic.\\
\indent Current models describing biological processes are deterministic, meaning they do not accurately reflect the randomness of complex systems over long periods of time. Dr. Motsch, a professor of applied biology and statistics in the School of Mathematical and Statistical Sciences, and I will study the same complex systems with the use of Brownian motion in order to obtain more accurate predictions of the behavior of stochastic complex systems. Brownian motion, originally studied by the botanist Robert Brown in the 19th century and popularized by Einstein in the early 20th century, describes motion that behaves stochastically. For instance, if a particle suspended in water were acted upon randomly by other particles, its behavior would be described as Brownian. My research with Dr. Motsch will seek to demonstrate that Brownian motion will accurately model stochastically complex systems and allow us to invent frameworks that derive meaningful insights into the future.\\
\indent In order to quantify the accuracy of our approach, we will be using a tool in mathematics called the Wasserstein Distance, which quantifies the distance between functions or data sets, and thereby provides us insight into the improvement that our approach to modeling stochastic systems offers. We aim to collaborate with
Dr. Pedro Lowenstein, a professor of Neurosurgery at the University of Michigan to obtain data on cancer growth in a mouse, and with this data, we will use the Wasserstein Distance to compute the distance between the true biological data, the predictions made by the Brownian motion approach, and the predictions made by the deterministic approach. We seek to understand how the number of particles used in the Brownian motion approach affects the accuracy of our approximation of the biological data. More specifically, we will propose an algorithm that will determine the number of particles necessary to get within a desired error using the Brownian motion approach. \\
\indent During a summer research project, completed under Dr. Anne Gelb in 2014, I studied the use of statistical algorithms to improve Fourier edge detection in the presence of corrupted data. Specifically, Dr. Gelb and I derived an algorithm for reconstructing MRI images that is 166$\%$ more accurate than previous methods available to the medical community. By working with Dr. Gelb, I developed an interest in using statistical tools to study relevant problems in biology. During the summer of 2016 I participated in two summer research programs that provided me with the interest and background necessary to successfully complete my Origins Project research. My first project was at the University of California, Santa Barbara where I took a series of courses on differential equations in random media, biological modeling of living systems, and statistical approximations of biological processes. Through conversations with professors and graduate students from Princeton, Stanford, UC Berkeley and others, I learned about cutting-edge research at the intersection of biology and statistics and will apply these techniques in my Origins Project research.\\
\indent In my second project, I was selected as one of three from over 300 undergraduates to participate in a research experience at San Diego State University. There I studied the convergence behavior of statistical algorithms used in linear regression models. Specifically, my advisor and I proposed the first algorithm for concrete convergence bounds for the error of a Gibbs sampler in a Bayesian linear regression model. These results will be submitted for publication in October. This research experience taught me the tools to complete statistical convergence analysis, and I will use similar tools to study the convergence behavior of the approximation to complex systems using Brownian motion. \\
\indent To successfully complete this research, I will need an advanced grasp of real analysis, probability theory, and numerical analysis. . I have taken a course on numerical methods with Dr. Motsch, and I have enrolled in graduate courses in real analysis, distribution theory, and stochastic processes. In these courses I will obtain the tools necessary to analyze papers relating to my work with Dr Motsch, as well as make meaningful contributions to the field. In the Spring I will finish the graduate sequence in real analysis and probability theory, creating a solid theoretical foundation that I will use in my research.\\
\indent Developing accurate statistical algorithms for understanding stochastic complex systems is one of the largest problems in modern biological sciences, and the opportunity to study this type of problem has been my main focus during my undergraduate career. In addition to contributing to the field of computational biology, I hope to motivate theoretical mathematicians and statisticians to solve problems of interest in applied fields, leading to more interactions and discussions between the previously distant groups of researchers. Indeed, the Wasserstein distance might also be successfully applied to problems in computer science and chemistry. In addition to significantly adding to my knowledge in statistics and computational mathematics, I hope this research project will start a larger discourse on the use of statistical algorithms, specifically Brownian motion, in biology, chemistry, and other applied fields. \\
\indent The Origins Project represents an avenue towards galvanizing an interest among pure mathematicians and statisticians in using theoretical tools to study problems of interest in biology. With the Origins Project’s Undergraduate Research Scholarship, I will have the opportunity to demonstrate the utility of mathematics and statistics in computational biology. By improving the accuracy of approximations in biology, which studies the development of infectious diseases and our place in the universe, my research will shed light on the origin and future of humanity.







\end{document}